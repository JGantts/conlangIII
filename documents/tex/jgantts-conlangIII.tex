%& -output-directory=./pdfout -aux-directory=./temp

\documentclass[11pt]{article}

% -----------------------------
% Packages
% -----------------------------

% For advanced font selection
\usepackage[linguistics]{forest}
\usepackage{linguex}

\usepackage{fontspec}
\usepackage{amssymb}

% For colorful and modern layouts
\usepackage{xcolor}
\usepackage{geometry}
\usepackage{graphicx}
\usepackage{tikz}
\usepackage{titlesec}
\usepackage{parskip} % Adds space between paragraphs
\usepackage{hyperref}
\usepackage{fancyhdr}
\usepackage{marginnote} % For margin notes
\usepackage{tikz-qtree}
\usepackage{float}
\usepackage{subcaption}
\usepackage{multirow}

\usepackage[block]{leipzig}
\usepackage{gb4e}
\usepackage{typgloss}

\usepackage{amsmath}
\usetikzlibrary{positioning}
\usepackage{array}

\usepackage{unicode-math}


% -----------------------------
% Theme Colors
% -----------------------------

\definecolor{primary}{RGB}{47, 128, 190}    % Muted Blue
\definecolor{secondary}{RGB}{192, 57, 43}   % Deep Red
\definecolor{accent}{RGB}{39, 174, 96}      % Cool Green
\definecolor{background}{RGB}{255, 255, 255} % Softer Light Gray
\definecolor{link}{RGB}{128, 0, 128}       % Regal Purple

% -----------------------------
% Page Geometry
% -----------------------------

\geometry{
    a4paper,
    left=1in,
    right=1in,
    top=1in,
    bottom=1in
}

% -----------------------------
% Fonts
% -----------------------------

\setmainfont{NotoSansDisplay}[
    Path=fonts/,
    Extension=.ttf,
    UprightFont=*-Regular,
    BoldFont=*-Bold,
    ItalicFont=*-Italic,
    BoldItalicFont=*-BoldItalic,
]

\newfontfamily{\textthin}{NotoSansDisplay}[
    Path=fonts/,
    Extension=.ttf,
    UprightFont=*-Thin,
    BoldFont=*-Light,
    ItalicFont=*-ThinItalic,
    BoldFont=*-LightItalic,
]

\newfontfamily{\textserif}{NotoSerifDisplay}[
    Path=fonts/,
    Extension=.ttf,
    UprightFont=*-Regular,
    BoldFont=*-Bold,
    ItalicFont=*-Italic,
    BoldItalicFont=*-BoldItalic,
]

\newfontfamily{\textserifthin}{NotoSerifDisplay}[
    Path=fonts/,
    Extension=.ttf,
    UprightFont=*-Thin,
    BoldFont=*-Light,
    ItalicFont=*-ThinItalic,
    BoldFont=*-LightItalic,
]

% -----------------------------
% Section Formatting
% -----------------------------

\titleformat{\section}
  {\color{primary}\normalfont\Large\bfseries}
  {\thesection}{1em}{}

\titleformat{\subsection}
  {\color{secondary}\normalfont\large\bfseries}
  {\thesubsection}{1em}{}

\titleformat{\subsubsection}
  {\color{accent}\normalfont\normalsize\bfseries}
  {\thesubsubsection}{1em}{}

% -----------------------------
% Header and Footer
% -----------------------------

\pagestyle{fancy}
\fancyhf{}
\fancyhead[L]{\textcolor{primary}{\leftmark}}
\fancyhead[R]{\textcolor{secondary}{\thepage}}
\fancyfoot[C]{\textcolor{accent}{JGantts Conlang III}}

% -----------------------------
% Hyperlinks
% -----------------------------

\hypersetup{
    colorlinks=true,
    linkcolor=link,
    urlcolor=accent,
    citecolor=secondary
}

% -----------------------------
% Title Styling
% -----------------------------

\newcommand{\customtitle}[1]{
    \begin{center}
        \vspace*{2cm}
        {\Huge\bfseries\color{primary} #1}\\
        \vspace{0.5cm}
        {\Large Kovyalo}\\
        \vspace{1cm}
    \end{center}
    \vspace{1cm}
    \hrule
    \vspace{1cm}
}

% -----------------------------
% Example Reference
% -----------------------------

\newcommand{\exref}[1]{(\ref{#1})}

% -----------------------------
% Tree Decoration
% -----------------------------

\tikzset{
  my colour/.style={text=#1},
  my tree/.style={
    font=\textthin,
  },
  my leaf/.style={
  },
}
\forestset{
  every leaf node/.style={
    if n children=0{#1}{}
  },
  every tree node/.style={
    if n children=0{}{#1}
  },
}

\newcommand{\leafnode}[2]{\textthin{#2} \\ #1 }

% -----------------------------
% Gloss Terms
% -----------------------------

\newleipzig{vbz}{vbz}{verbalizer}
\newleipzig{c}{c}{chaos class}
\newleipzig{l}{l}{life class}
\newleipzig{o}{o}{order class}
\newleipzig{d}{d}{death class}
\newleipzig{hab}{hab}{habitual}
\newleipzig{vol}{vol}{voluntary}
\newleipzig{nvol}{nvol}{involuntary}
\newleipzig{pprt}{pprt}{past participle}
\newleipzig{aprt}{aprt}{active participle}
\newleipzig{sup}{sup}{superessive}
\newleipzig{pert}{pert}{pertingent}
\newleipzig{dummy}{dummy}{dummy}
\newleipzig{comm}{comm}{communicative}
\newleipzig{dir}{dir}{direct}
\newleipzig{idr}{idr}{indirect}

\makeglossaries

\glsdisablehyper
\renewcommand{\leipzigfont}[1]{\color{primary}{\textsc{#1}}}

% -----------------------------
% Document
% -----------------------------

\begin{document}
\pagecolor{background}

% Custom Title
\customtitle{JGantts Conlang III}

% Introduction Section
\section{Intro}
This is my third major conlang attempt since 2020-ish.
The "core" setting a far-future world where a generation ship landed,
but about 6 years after landing, there was a coordinated act of sabotage.
The sabotage destroyed all power generation stations, a significant portion
of the power grid, and all (3D) printers capable of printing circuitry.

\section{Phonology}

\begin{table}[H]
\begin{center}
\caption{IPA Phoneme}
\label{table-phonemes-ipa}
\begin{tabular}{l l l l l l l l l l c c}
 m & mʷ & n & & & &   &    &   & & i & u \\
 p & pʷ & t & & & & k & kʷ & ʔ & & e & o \\
 b & bʷ & d & & & & g & gʷ &   & & a & ɒ \\
 f & & s & ʃ & t͡ʃ & t͡ʃʷ & x & xʷ & & &  \\
 & & z & ʒ & d͡ʒ & ʒʷ & & & & & \\
 w & & l & ɹ & j & & & & \\
\end{tabular}
\end{center}
\end{table}

\begin{table}[H]
\begin{center}
\caption{Latin}
\label{table-phonemes-latin}
\begin{tabular}{l l l l l l l l l l c c}
 m & mu & n & & & &   &    &    & & i & v \\
 p & pu & t & & & & k & qu & tt & & e & o \\
 b & bu & d & & & & g & gu &    & & æ & a \\
 f & & s & c & tc & tcu & h & hu & & & \\
 & & z & ç & dç & çu & & & & & \\
 u & & l & r & y & & & & \\
\end{tabular}
\end{center}
\end{table}

\paragraph{Labialization}
Labilized consonants are \textit{phonemeically} forbiden before
rounded/back\footnote{
Backness corresponds exactly with roundness.
All back vowels are round and all round vowels are back.
}
vowels.
\textit{Phonetically} all consonants before rounded vowels are labialized.
This is revealed only in allohphones;
that is to say, in production.
In perception (and orthography) they are unlabialized consonants.
Example \exref{ex:phono-labialized} shows both valid and invalid words.

\begin{exe}
\ex \label{ex:phono-labialized}
\begin{xlist}
\ex [] {$\langle$buine$\rangle$ /bʷine/}
\ex [] {$\langle$buæne$\rangle$ /bʷane/}
\ex [*] {$\langle$buone$\rangle$ /bʷone/}
\end{xlist}
\end{exe}

\paragraph{Valid Syllables Structure}
This chart does not account for the labialized consonant/rounded vowel ban.

\begin{forest}
  for tree={}
[ Syllable
	[ Onset
   		[ C^1
			[
$\begin{bmatrix}
\textrm{C}
\end{bmatrix}$
			]
		]
		[ C^2 C^3
			[
$\begin{bmatrix}
\textrm{s} & \textrm{ʃ}
\end{bmatrix}$
$\begin{bmatrix}
\textrm{m} & \textrm{n}\\
\textrm{p} & \textrm{pʷ}\\
\textrm{k} & \textrm{kʷ}
\end{bmatrix}$
			 ]
		 ]
	]
	[ Rhyme 
		[ Nucleus 
			[ V
				[
$\begin{bmatrix}
\textrm{i} & \textrm{u}\\
\textrm{e} & \textrm{o}\\
\textrm{a} & \textrm{ɒ}
\end{bmatrix}$
			 	]
			 ]
		]
		[ Coda
   			[ C^4
				[
$\begin{bmatrix}
\textrm{C}
\end{bmatrix}$
				]
			]
			[C^2 C^5   %(n|ʃ|t͡ʃ|ʒ|d͡ʒ|k|l|p|s|t|ʃp|ʃk|sp|sk)
				[
$\begin{bmatrix}
\textrm{s} & \textrm{ʃ}
\end{bmatrix}$
$\begin{bmatrix}
\textrm{p} & \textrm{k} \\
\end{bmatrix}$
%
% n ʃ t͡ʃ ʒ d͡ʒ k l p s t ʃp ʃk sp sk
%
				]
			]
		]
	]
]
\end{forest}

Valid syllables include 
$\langle$spisk$\rangle$ /spisk/,
$\langle$tan$\rangle$ /tan/,
and $\langle$ckuik$\rangle$ /ʃkʷik/.

C^1 is any consonant save for $\langle$tt$\rangle$ /ʔ/.
C^2 is any consonant save for $\langle$m r u y$\rangle$ /m ɹ w j/.

\newpage
\section{Grammar}

\subsection{Intro to Grammar}

\paragraph{Basic Sentences}
The basic sentence is SVO.
Subject (Nominative) is in the Direct Case.
Object (Accusative) is in the Indirect Case.

\begin{exe}
\ex \label{basicsentence}
Quedi ogacmat bicket.
\gll
qu-  edi og      acmat b-   icket\\
DIR  1SG PST.PFV eat   IDR- cake\\
\trans 
    \textit{'I ate (some) cake.'}\\
\end{exe}

\paragraph{Ditransitive Sentences}
Dative is in the Indirect Case as well, proceeded by $\langle$iga$\rangle$.

\begin{exe}
\ex \label{basicsentence}
Quepona ogattot iga bedi bicket.
\gll
qu-  epona  og      attot iga b-   edi b-   icket\\
DIR  3SG/DU PST.PFV give  DAT IDR- 1SG IDR- cake\\
\trans 
    \textit{'She gave me (some) cake.'}\\
\end{exe}


\subsection{Nouns}

\paragraph{Classes}
There are four noun classes,
\textbf{Chaos}, \textbf{Life}, \textbf{Order}, and \textbf{Death}.
They are the four classes of magic used in the conworld the langauge is set in.
There are patterns in the classes, but overall they are arbitrary.
The exception being Life class;
it holds all animate nouns.
Table \ref{table-cases} shows the case prefixes.

\begin{table}[H]
\begin{center}
\caption{Case Prefixes}
\label{table-cases}
\begin{tabular}{llllll}
                            &           & \textbf{Chaos} & \textbf{Life} & \textbf{Order} & \textbf{Death} \\
\hline\hline
\multirow{2}{*}{DIR}        & SG/DU     & tc             & k, qu         & h              & r              \\
                            & PL        & m, mu          & ik, iqu       & dç             & g              \\
\hline
\multirow{2}{*}{IDR}                 & SG/DU     & u, y           & b             & f              & s              \\
                            & PL        & l              & ib            & if             & z              \\
\hline
\multirow{2}{*}{IDR (VOL)}  & SG/DU     &                & p, pu         &                &                \\
                            & PL        &                & ip, ipu       &                &                \\
\hline
\multirow{2}{*}{OBL}        & SG/DU     & tatt           & oh            & g, gu          & zit            \\
                            & PL        & datt           & ih            & ag, agu        & izit          
\end{tabular}
\end{center}
\end{table}

\paragraph{Cases}
Direct and Indirect correspond to Nominative and Accusative respectively for most verbs.
See subsection \ref{sec:voluntary} "Voluntary Accusative" for more information on the exceptions.

\paragraph{Number}
There are two markers for number. SG/DU marks the Singular/Dual case;
for singles and pairs. PL, Plural, marks anything in threes or above.

\newpage

\subsection{Voluntary Accusative} \label{sec:voluntary}

\paragraph{Intro}
The vast majority of verbs follow the NOM/ACC alignment.
However, there is a class of verbs, called Voluntary Verbs, which follow ERG/ABS alignment.
The Voluntary Indirect is paired with the Unmarked Indirect.
These two markers are used to switch the meaning of the verb
in ways that would typically be different lexemes in English.

\paragraph{Hear vs Listen — vbino \textit{(Sing)}}
Example \exref{voluntary} shows us how Voluntary Indirect can be used.
Ex. \exref{voluntary:invol} features the Unmarked (involuntary) Indirect to mean "hear".
Ex. \exref{voluntary:vol} meanwhile, shows the Voluntary Indirect to mean "listen".

\begin{exe}
\ex \label{voluntary}
\begin{xlist}
\ex \label{voluntary:invol}
Iga bedi ogvbino metil.
\gll
iga  b-         edi og-     vbino     m-       etil \\
COMM IDR.SG/DU- 1SG PST.PFV sing DIR.PL- birds \\
\trans 
    {\textthin{}'The birds sang to me.' \textit{(passive patient)}}\\
    \textit{'I heard the birds singing.'}

\ex \label{voluntary:vol}
Iga puedi ogvbino metil.
\gll
iga  pu-            edi og-     vbino     m-       etil \\
COMM IDR.SG/DU.VOL- 1SG PST.PFV sing DIR.PL- birds \\
\trans 
    {\textthin{}'The birds sang to me.' \textit{(active patient)}}\\
    \textit{'I listened to the birds singing.'}
\end{xlist}
\end{exe}

\paragraph{Go vs Fall — ormir \textit{(Lower)}}
Here a dummy pronoun is used when there is no obvious Actor.
The dummy gets the Nominative marker.
In the case of \textbf{ormir} \textit{(lower)} the Nominative Chaos marker is used.
The verbs are grouped by which class of marker they use.

\begin{exe}
\ex \label{fall}
\begin{xlist}
\ex \label{fall:invol}
Tcitti bedi ogormir uilompi.
\gll
tc-       itti  b-         edi og-     ormir uilompi\\
DIR.C.SG/DU DUMMY IDR.SG/DU- 1SG PST.PFV lower  downstairs\\
\trans 
    {\textthin{}'It tossed me downstairs.'}\\
    \textit{'I fell downstairs.'}\\

\ex \label{fall:vol}
Tcitti puedi ogormir uilompi.
\gll
tc-       itti  pu-            edi og-     ormir    uilompi\\
DIR.C.SG/DU DUMMY IDR.SG/DU.VOL- 1SG PST.PFV lower downstairs\\
\trans 
    {\textthin{}'It lowered me downstairs.'}\\
    \textit{'I went downstairs.'}\\
\end{xlist}
\end{exe}



\begin{table}[H]
\begin{tabular}{ m{1em} m{5em} | m{5em} | m{5em}| m{10em} }
& Lexeme & INVOL & VOL & English Approx \\
\hline\hline
\multicolumn{5}{l}{ \textbf{Chaos} } \\
\hline
& ormir & fall & descend & lower \\ 
& amuak & drift & go & put \\ 
&       & stumble & run & put \\
&       & sink & dive & sink (something)\\ 

& aspabe & be hit & catch & throw (at and hit) \\ 

\hline
\multicolumn{4}{l}{ \textbf{Life} } \\
\hline
&        & see & look & show (self)\\ 
& esnvbe & hear & listen & tell\\ 
& vbino  & hear & listen & sing\\ 
       
\hline
\multicolumn{4}{l}{ \textbf{Order} } \\
\hline
&       & remember & think & remind \\ 
&       & realize & decide & pursuade\\ 
       
\hline
\multicolumn{4}{l}{ \textbf{Death} } \\
\hline
& ackak  & shatter & break & break \\
&       & drop & throw & take\\  

\end{tabular}
\end{table}

\paragraph{Be Hit vs Catch – aspabe \textit{(Throw)}}
\begin{exe}
\ex \label{fall}
\begin{xlist}
\ex \label{fall:invol}
Tcitti iga bedi ogaspabe fipa.
\gll
tc-         itti  iga  b-         edi og-     aspabe f-    ipa\\
DIR.C.SG/DU DUMMY COMM IDR.SG/DU- 1SG PST.PFV throw  ACC.O rock\\
\trans 
    {\textthin{}'It threw a rock at me.'}\\
    \textit{'A rock hit me.'}\\

\ex \label{fall:vol}
Tcitti iga puedi ogaspabe fipa.
\gll
tc-         itti  iga  pu-            edi og-     aspabe f-    ipa\\
DIR.C.SG/DU DUMMY COMM IDR.SG/DU.VOL- 1SG PST.PFV throw  ACC.O rock\\
\trans 
    {\textthin{}'It threw a rock to me.'}\\
    \textit{'I caught a rock.'}\\
\end{xlist}
\end{exe}


\newpage

\paragraph{Shatter vs Break – ackak (ACC/NOM)}
\begin{exe}
\ex \label{break:S}
\begin{xlist}
\ex \label{fall:invol}
Reno ogackak.
\gll
r-  eno    og-     ackak\\
DIR window PST.PFV break\\
\trans 
    \textit{'The window broke.'}\\

\ex \label{break:AN}
Reno bedi ogackak.
\gll
r-  eno    b-   edi og-     ackak\\
DIR window IDR- 1SG PST.PFV break\\
\trans 
    \textit{'I broke the window. (invol)'}\\

\ex \label{break:EA}
Reno puedi ogackak.
\gll
r-  eno    pu-      edi og-     ackak\\
DIR window IDR.VOL- 1SG PST.PFV break\\
\trans
    \textit{'I broke the window. (vol)'}\\
\end{xlist}
\end{exe}

\paragraph{Eat – acmat (ERG/ABS)}
\begin{exe}
\ex \label{eat:S}
\begin{xlist}
\ex \label{fall:invol}
Quedi ogacmat.
\gll
qu-  edi og      acmat\\
DIR  1SG PST.PFV eat\\
\trans 
    \textit{'I ate.'}\\

\ex \label{eat:AN}
Quedi ogacmat bicket.
\gll
qu-  edi og      acmat b-   icket\\
DIR  1SG PST.PFV eat   IDR- cake\\
\trans 
    \textit{'I ate (some) cake.'}\\
\end{xlist}
\end{exe}

\newpage

\begin{exe}
\ex \label{ex:catwindow}
Tcipo ogitcuido bino mvho gueno, vrotizvlo uitto otemvsti yvno gueno.
\gll
tc-         ipo og-     itcuido bino {mvho gueno} vr-   ot-      izvlo u-          itto ot-      emvsti yvno gueno\\
NOM-C.SG/DU cat PST.PFV perch   SUP  windowsill   APRT- PST.IPFV watch C.ACC.SG/DU rain PST.IPFV flow   PERT glass\\
\trans `The cat perched on the windowsill, watching the rain cascade down the glass.'\\
\end{exe}

\scalebox{1}{
\begin{forest}
  for tree={
    every leaf node={my leaf},
    every tree node={my tree},
    parent anchor=south,
    child anchor=north,
    edge=primary
  }
[ VP
    [ NP
        [ \leafnode{tcipo}{cat.\Nom{}} ]
    ]
    [ V'
        [V'
        	[VP
        		[ V'
					[V'
    					[ V
    	        			[ \leafnode{ogitcuido}{perched.\Pst{}.\Pfv{}} ]	
    					]
            		]
        	        [PP [on the windowsill,roof ] ]
	        	]
            ] 
        ]
	        [CP 
	        	[watching...,roof]
			]
    ]
 ]
\end{forest}
}

Example \exref{ex:catwindow} shows us how Participles can be used.

\newpage

\section{Fun}
\begin{exe}
\ex \label{ex:whisper1}
Melibe dçaraqueçv uinalo iga puina.
\gll
m-        elibe dç-  ar-     aqueçv  u-     inalo  iga pu-        ina\\
NOM.C.PL- star  HAB- VBZ.IPFV- whisper ACC.O- secret to  ACC.L.VOL- 3PL\\
\trans `The stars whisper secrets to those who listen.'\\
\end{exe}

\scalebox{0.75}{
\begin{forest}
  for tree={
    every leaf node={my leaf},
    every tree node={my tree},
    parent anchor=south,
    child anchor=north,
    edge=primary
  }
[ IP
    [ NP
        [ SPEC
            [ \leafnode{m}{\Nom{}.\C{}.\Pl{}} ]
        ]
        [ N
            [ \leafnode{elibe}{star(s)} ]
        ]
    ]
    [ I'
        [ I
	        [\leafnode{dç}{\Hab{}} ]
        ]
        [ IP
            [ I
              	[\leafnode{ar}{\Vbz{}.\Ipfv{}} ]
			]
			[ VP
                [ V'
                    [ V 
                    	[\leafnode{aqueçv}{whisper} ]
                    ]
                    [ NP
                        [ SPEC
    	                    [\leafnode{u}{\Acc{}.\O{}.\Pl{}} ]
                        ]
                        [ N
                        	[\leafnode{inalo}{secret(s)} ]
                        ]
                    ]
                ]
                [ PP
                    [ P
                        [ \leafnode{iga}{\All{}} ]
                    ]
                    [ NP
                        [ SPEC
                        	[ \leafnode{pu}{\Acc{}.\L{}.\Vol{}} ]
                        ]
                        [ N
                            [ \leafnode{ina}{\Third{}\Pl{}} ]
                        ]
                    ]
                ]
            ]
        ]
    ]
]
\end{forest}
}


\newpage

\setglossarysection{section}
\printglosses

\end{document}
